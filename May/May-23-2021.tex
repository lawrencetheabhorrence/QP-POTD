\documentclass[answers]{exam}
\usepackage{amsmath}
\usepackage{mathtools}
\DeclarePairedDelimiter\ceil{\lceil}{\rceil}
\DeclarePairedDelimiter\floor{\lfloor}{\rfloor}
\title{Problem of the Day - 3}
\date{May 23, 2021}
\renewcommand{\solutiontitle}{\noindent\par\noindent}

\begin{document}
	\begin{questions}
		\question
		Given the equation
		\begin{equation*}
			(x^2 - 3x -2)^2 -3(x^2 -3x - 2) - 2 - x = 0
		\end{equation*}
		prove that the roots of the equation $x^2 - 4x - 2 = 0$ are roots of the initial equation and find all real roots of the equation.
		
		\begin{solution}
			\textbf{Solution 1: } Let
			\begin{equation*}f(x) = x^2 - 4x - 2\end{equation*}
			and $r_1, r_2$ be the roots of $f(x)$. Thus we can express the first equation as
			\begin{equation*}
				(f(x) + x)^2 -3(f(x) + x) - 2 - x = 0
			\end{equation*}
			Substituting $r_1$ into this equation, knowing that $f(r_1) = 0$ (you can also use $r_2$).
			\begin{equation*}
				r_1^2 - 4r_1 - 2 = 0 \\
			\end{equation*}
			This implies the first equation is true if this new equation is also true. Since
			\begin{equation*}
				f(r_1) = r_1^2 - 4r_1 - 2 = 0
			\end{equation*}
			Then the first given equation must be true for $x = r_1$. This is the
      same for $r_2$. \par

      \textbf{Solution 2 (credits to n'ada):} Let
      \begin{equation*}
        f(x) = x^2 - 3x - 2
      \end{equation*}
      Thus if we can express the first equation as looking for an $x$ such that
      \begin{equation*}
        f(f(x)) - x = 0
      \end{equation*}
      From $x^2 - 4x - 2 = 0$ we have
      \begin{gather*}
        f(x) - x = 0 \\
        f(x) = x
      \end{gather*}
      Thus
      \begin{equation*}
        f(f(x)) - x = f(x) - x = 0
      \end{equation*}
      Thus the roots to the first equation are the roots to $f(x) - x = 0$
      which is $x^2 - 4x - 2 = 0$.
		\end{solution}
	\end{questions}
\end{document}
