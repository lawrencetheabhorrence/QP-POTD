\documentclass[answers]{exam}
\usepackage{amsmath}
\usepackage{mathtools}
\DeclarePairedDelimiter\ceil{\lceil}{\rceil}
\DeclarePairedDelimiter\floor{\lfloor}{\rfloor}
\title{Problem of the Day - 2}
\date{May 18, 2021}

\begin{document}
	\maketitle
	\begin{questions}
		\question
		Given that $\log_{10}2 \approx 0.3010$, how many digits are in $5^{44}$?
		
		\begin{solution}
			The number of digits of a number $x$ is $\floor*{x} + 1$ (Try it with a few examples).
			\begin{equation*}
			\log 5^{44} = 44 \log 5 = 44 \log (10/2) = 44(1 - 0.3010) \approx 30
			\end{equation*}
			Thus $5^{44}$ has 31 digits.
		\end{solution}
	
		\question
		If $60^a = 3$ and $60^b = 5$, then find $12^{\frac{1 - a - b}{2-2b}}$
		
		\begin{solution}
			From the given, we can easily find a and b.
			\begin{gather*}
			a = \log_{60} 3 \\
			b = \log_{60} 5 \\
			\end{gather*}
			Thus, evaluate the numerator and denominator of the exponent of 12.
			\begin{gather}
			1 - a - b = \log_{60}60 - \log_{60} 3 - \log_{60} 5 = \log_{60}(60 / 15) = \log_{60} 4 \\
			2 - 2b = 2(1 - b) = 2\left (\log_{60}60 - \log_{60}5 \right) = 2\log_{60} 12
			\end{gather}
			Since these are exponents of 12, it would be better if we convert (1) and (2) to base 12. \\
			(Remember that $12^{\log_{12}x} = x$)
			\begin{gather*}
			\log_{60} 4 = \frac{\log_{12} 4}{\log_{12} 60} \\
			\log_{60} 12 = \frac{\log_{12} 12}{\log_{12} 60}
			\end{gather*}
			Thus,
			\begin{equation*}
			\frac{1 - a - b}{2 - 2b} = \frac{\log_{12} 4}{2}
			\end{equation*}
			Now we can evaluate the expression.
			\begin{align*}
			12^{\frac{1 - a - b}{2 - 2b}} &= 12^{\frac{\log_{12} 4}{2}} \\
			&= \sqrt{12^{\log_{12} 4}} = \sqrt{4} = 2
			\end{align*}
		\end{solution} 
	\end{questions}
\end{document}
